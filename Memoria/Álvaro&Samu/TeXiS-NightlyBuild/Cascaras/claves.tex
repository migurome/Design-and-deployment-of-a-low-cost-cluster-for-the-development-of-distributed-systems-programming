%---------------------------------------------------------------------
%
%                      claves.tex
%
%---------------------------------------------------------------------
%
% Contiene el cap�tulo de las palabras clave en castellano.
%
% Se crea como un cap�tulo sin numeraci�n.
%
%---------------------------------------------------------------------

\chapter{Palabras clave}
\cabeceraEspecial{Palabras clave}

\begin{FraseCelebre}
\begin{Frase}
Las palabras son como monedas, que una vale por muchas como muchas no valen por una.
\end{Frase}
\begin{Fuente}
Francisco de Quevedo, Escritor espa�ol.
\end{Fuente}
\end{FraseCelebre}

%\partTitle{Palabras clave}

%\partDesc{
\begin{itemize}
	\item Algoritmo evolutivo
	\item Algoritmo gen\'etico
	\item Mazmorra
	\item Inteligencia artificial
	\item Grafo
	\item Programaci\'on gen\'etica
	\item \'Arbol
	\item Selecci\'on
	\item Mutaci\'on
	\item Cruce
\end{itemize}
%}

%\makepart
% Variable local para emacs, para  que encuentre el fichero maestro de
% compilaci�n y funcionen mejor algunas teclas r�pidas de AucTeX
%%%
%%% Local Variables:
%%% mode: latex
%%% TeX-master: "../Tesis.tex"
%%% End:
