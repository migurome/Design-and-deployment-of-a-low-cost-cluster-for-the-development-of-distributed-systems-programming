%---------------------------------------------------------------------
%
%                      abstract.tex
%
%---------------------------------------------------------------------
%
% Contiene el cap�tulo del resumen en ingl�s.
%
% Se crea como un cap�tulo sin numeraci�n.
%
%---------------------------------------------------------------------

\chapter{Abstract}
\cabeceraEspecial{Abstract}

Evolutionary computing is a branch of AI that includes a set of techniques that, through the simulation of natural processes and genetics, are used to solving complex problems of search and learning. These problems can be solved through EAs.\par
The aim of this project is to automate tedious process involved in a videogame using evolutionary techniques and design a simple video game using them. For this purpose an automatic map generator is used and some AIs will be generated representing the enemies in the game.\par
The maps will be generated through a GA. The data structure that has been considered to represent the video game maps is a graph. The rooms would be represented by the nodes, while the hallways that unite rooms would be the edges.\par
On the other hand, the AI of the enemies, will be managed through genetic programming. This algorithm is in charge of the behaviour, both before and after of detecting the player.\par
We have made a framework for genetic programming which allows to experiment and test different techniques for the AI generation, allowing to modify and adjust any parameter desired involved in the process.\par
\endinput
% Variable local para emacs, para  que encuentre el fichero maestro de
% compilaci�n y funcionen mejor algunas teclas r�pidas de AucTeX
%%%
%%% Local Variables:
%%% mode: latex
%%% TeX-master: "../Tesis.tex"
%%% End:
