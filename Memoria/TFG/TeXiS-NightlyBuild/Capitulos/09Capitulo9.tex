%---------------------------------------------------------------------
%
%                          Capítulo 9
%
%---------------------------------------------------------------------

\chapter{Conclusiones y trabajo futuro}
\begin{FraseCelebre}
\begin{Frase}
A fin de cuentas, todo es un chiste.
\end{Frase}
\begin{Fuente}
Charles Chaplin, Actor y director brit\'anico.
\end{Fuente}
\end{FraseCelebre}

%\begin{resumen}
%...
%\end{resumen}

\section{Conclusiones en la generaci\'on de mazmorras}
Tras el an\'alisis requerido para conseguir un algoritmo gen\'etico que fuese capaz de generar mazmorras acordes a nuestras expectativas, nos parece importante concluir si el trabajo realizado puede ser considerado \'util para su implementaci\'on real en un videojuego.\par
En primer lugar, podemos decir que el resultado obtenido en este apartado ha satisfecho con creces nuestras expectativas iniciales, no s\'olo desde el punto de vista de la calidad de las mazmorras, sino tambi\'en en lo referente a tiempos de ejecuci\'on.\par
Por otro lado, es cierto que hallar una configuraci\'on adecuada del AG, en especial la funci\'on de fitness, requiere de mucho tiempo, dedicaci\'on y pruebas que deben ser consideradas antes de optar por el uso de t\'ecnicas evolutivas.\par
Dicho esto, creemos que este proceso creativo y de desarrollo depende en gran medida de la experiencia previa y en nuestro caso particular, la configuraci\'on \'optima de los par\'ametros para afinar la evaluaci\'on de los mapas ha supuesto un gran esfuerzo y ha requerido una cantidad de tiempo considerable.\par
Consideramos que la t\'ecnica evolutiva utilizada para la generaci\'on de mapas nos ha servido para experimentar con una una funci\'on de fitness calibrada y ajustada en funci\'on de los par\'ametros seleccionados como relevantes, para definir una representaci\'on novedosa de los individuos o mapas, y para experimentar con todos los operadores gen\'eticos adecuados a esta representaci\'on.\par
\section{Conclusiones en la generaci\'on de inteligencias artificiales}
En la generaci\'on de las IAs ha sido donde m\'as nos ha costado obtener resultados convincentes. Al principio cometimos el error de ser demasiado ambiciosos, como se ha visto en las m\'ultiples aproximaciones para la funci\'on de fitness. Sin embargo, despu\'es de todas las pruebas y resultados, podemos concluir que es viable obtener inteligencias artificiales lo suficientemente buenas como para incluirlas directamente en un videojuego. Por supuesto, tambi\'en hay que tener en cuenta que cuanto m\'as complejo sea el videojuego, m\'as complicado es conseguir una funci\'on de evaluaci\'on adecuada, ya sea por la cantidad de operaciones que habr\'ia que inclu\'ir as\'i como la complejidad de la simulaci\'on de la IA del jugador.\par
Sin embargo, somos muy optimistas con los resultados que hemos obtenido. Hemos conseguido obtener unas IAs lo suficientemente buenas como para incluirlas en el videojuego. Por tanto, podemos concluir que este trabajo ser\'ia tanto viable para generar un videojuego de forma real y susceptible de incorporar muchas mejoras para mejorar el comportamiento general.\par
\section{Conclusiones generales}
Nuestra apreciaci\'on final es que trabajar con algoritmos evolutivos requiere muchas pruebas, conocerlos exhaustivamente e implementar operaciones y estructuras en ocasiones complejas.\par
Teniendo esto en cuenta, cabe destacar que hemos observado muy buenos resultados en ambas partes y con un tiempo limitado. Tenemos confianza en que, si tuvi\'eramos que volver a realizar algo similar, nos resultar\'ia m\'as sencillo y conseguir\'iamos resultados buenos en menos tiempo.\par
Creemos que las t\'ecnicas evolutivas, en comparaci\'on con las t\'ecnicas procedurales, presentan ventajas en cuanto a la versatilidad y posibilidad de reutilizaci\'on. Por otra parte, esta facilidad en el m\'etodo se opone al hecho de definir bien la estructura de los individuos, as\'i como encontrar una forma adecuada de evaluarlos.\par
Por tanto, podemos concluir que la utilizaci\'on de algoritmos evolutivos para la generaci\'on autom\'atica de contenidos para videojuegos es viable y \'util. Esto se ha podido apreciar con la prueba de concepto creada que, si bien es simple, sirve para demostrar nuestro enfoque.\par
\section{Trabajo futuro}
Como se ha insinuado en cap\'itulos anteriores, para nosotros ha sido un gran desaf\'io tener que desarrollar todo nuestro proyecto desde cero, empezando por el proceso creativo, con sus reuniones de \textit{brainstorming} hasta el dise\~no completo de una interfaz que, si bien es mejorable, cumple nuestras necesidades. Por este motivo y dada la naturaleza del proyecto, a\'un existen opciones y mejoras sobre nuestro trabajo. Consideramos que el bloque de generaci\'on de mazmorras es definitivo. Podr\'ia ser la base para crear nuevos tipos de mazmorras o incluso un generador parametrizado para distintos tipos de mapas y mazmorras.\par
Dentro de est\'a aplicaci\'on, podr\'ia estudiarse la inclusi\'on de t\'ecnicas de generaci\'on procedural combinadas con los propios algoritmos evolutivos, con la idea de encontrar soluciones a\'un mejores. Sin ser un desarrollo completo, hemos podido aplicar este enfoque h\'ibrido en la implementaci\'on del videojuego, combinando la generaci\'on de mazmorras mediante AG con un m\'etodo procedural encargado de rellenar cada sala con los elementos necesarios.\par
Como hemos querido hacer ver con la implementaci\'on del framework de pruebas para las IAs, nuestra intenci\'on es que puedan ampliarse las opciones para ser utilizadas en distintos tipos de videojuegos, manteniendo la base que proporcionamos.\par
Analizar el rendimiento de generaci\'on de IAs mediante gram\'aticas evolutivas, en lugar de programaci\'on gen\'etica, ser\'ia otro de los puntos que nos hubiese gustado analizar. Si bien el uso de gram\'aticas evolutivas es m\'as simple a efectos de implementaci\'on, su uso conlleva de una u otra forma traducir las expresiones obtenidas al formato que requiera la aplicaci\'on sobre la que se quieran usar. Por otro lado, con un mismo algoritmo se pueden hacer evolucionar diferentes gram\'aticas, pudiendo optar por gram\'aticas de grano fino, con terminales simples, similares a los que hemos utilizado, u otras de grano m\'as grueso, en las que los terminales representan acciones m\'as complejas, como estrategias completas. Estas estrategias pueden ser ofensivas, de evasi\'on, etc.\par
\section{Conclusions in dungeon generation:}
After the required analysis to achieve a genetic algorithm capable of generating dungeon as we expected them, we think it is important to conclude if the work done here can be considered useful for a real video game implementation.\par
In the first place, we can say that the results obtained in this part of the project have been greatly satisfactory. The dungeons generated are as we expected not only in the quality but in the time expended to generate them.\par
On the other hand, it is true that finding the proper configuration for a GA, specially the fitness function, can be hard and it requires a lot of dedication, work and tests. All of this should be kept in mind before opting for the use of evolutionary techniques.\par
All said, we believe that this creative and development process depends on previous experience highly, and, in our particular case, the consecution of optimal parameters was a big deal and we had to expend a lot of time and resources to get it.\par
Finally, We consider that this final fitness function allowed us to gain experience in genetic programming and genetic algorithms as well as give a new representation to maps and dungeons in video games.\par
\section{Conclusions in artificial intelligence generation:}
It is in this part where we had the biggest issues. In the first place we made the mistake of being too ambitious, as it has been seen in the multiple approximations to the fitness function we wrote before. However, after all the tests and results, we can conclude that it is viable to obtain good enough artificial intelligences to be included in a game. Of course, we all have to keep in mind that, the harder the game is, the harder to achieve a fitness function is.\par
Nevertheless, we are really optimistic with the results obtained. We finally got AIs good enough to be included in the video game. So, we can say that this work could be also viable to generate AIs for a real video game, and it is still susceptible of improvements to upgrade the general behaviour of the AIs.\par
\section{General conclusions:}
Our final appreciation is the following: Working with genetic algorithms requires a lot of previous testing, knowing them perfectly and implementing operations and data structures which are, sometimes, complex.\par
Keeping this in mind, it is important to point out that we have appreciated a lot of good results in both main objectives and with little time. We are confident in, with the proper amount of time, we could get something similar done easily.\par
We believe that evolutionary techniques, compared to procedural generation techniques, presents several advantages in versatility and reusability. On the contrary, this easy in the method it is opposed to the fact the programmer has to define properly the structure of the individuals, as well as founding a fine way of evaluating them. \par
So, we can conclude that utilize genetic algorithm for automatic generation of content in video games is viable and useful. This can be better of appreciated with the proof of concept provided in this project. Even though this proof of concept is simple, it is good enough to help us prove our point.\par
\section{Future work:}
As it has been insinuated in previous chapters, this project was a big deal for us, since we developed everything from scratch. We began with the creative process and ended up with all this. For this reason and the nature of the project, there are still options to develop and upgrades to be made. We consider the dungeon generation block is final. This part could be the foundations to develop new kinds of dungeons or even a parametric generator for several maps and dungeons.\par
Inside this app, it could be a matter of study the procedural generation techniques combined with GA. In this section we have started this hybrid approach in the video game, combining the dungeon generation with a simple procedural method which fills the rooms with the proper elements.\par
As we tried to put the focus on with the framework implementation for the AIs, our intention has always been to give a bunch of tools for game developers.\par
To analyze the AI generation using grammars instead of genetic programming is another work that can be done, since we did not have the time to do it. However, you have to keep in mind that, even the grammars are easier to implement, their use implies to develop a method to translate this grammars into the app they are meant to be used. On the other side, with the same algorithm and using grammars, this algorithm could potentially evolve several of them. Ones could be similar as the functions and nodes we used, and other a little bit more complex,  having terminals representing high level actions and complex strategies. This strategies could be offensive, defensive or whatever the developer wants.\par
% Variable local para emacs, para  que encuentre el fichero maestro de
% compilación y funcionen mejor algunas teclas rápidas de AucTeX
%%%
%%% Local Variables:
%%% mode: latex
%%% TeX-master: "../Tesis.tex"
%%% End:
