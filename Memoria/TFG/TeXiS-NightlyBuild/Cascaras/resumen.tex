%---------------------------------------------------------------------
%
%                      resumen.tex
%
%---------------------------------------------------------------------
%
% Contiene el cap�tulo del resumen.
%
% Se crea como un cap�tulo sin numeraci�n.
%
%---------------------------------------------------------------------

\chapter{Resumen}
\cabeceraEspecial{Resumen}

\begin{FraseCelebre}
\begin{Frase}
Cualquier tecnolog\'ia suficientemente avanzada es indistinguible de la magia.
\end{Frase}
\begin{Fuente}
Arthur C. Clarke (1917-2008) Escritor ingl\'es de ciencia ficci\'on.
\end{Fuente}
\end{FraseCelebre}

La computaci�n evolutiva es una rama de la IA que engloba un conjunto de t�cnicas que, a trav�s de la simulaci�n de procesos naturales bioinspirados, son utilizados para la resoluci�n de problemas complejos de b�squeda y aprendizaje.\par
Este trabajo presenta una serie de t�cnicas evolutivos aplicadas a la generaci�n autom�tica de contenidos en videojuegos. El objetivo de este Trabajo de Fin de Grado es automatizar procesos tediosos y repetitivos propios de un videojuego mediante el uso de estas t�cnicas  y utilizarlas para crear un videojuego simple. Para ello hemos dividido el trabajo en dos bloques principales: un generador de mapas sobre los que se desarrollar� el juego -formados por diferentes salas- y un generador de estrategias o Inteligencias Artificiales (IAs) para los enemigos contra los que se enfrenta el jugador en el videojuego.\par
Los mapas sobre los que se desarrolla el juego se generan utilizando un algoritmo evolutivo. La estructura de datos que se ha considerado utilizar para representar los mapas del videojuego es un grafo que representa el genotipo del individuo que haremos evolucionar. Las salas del mapa estar�an representadas mediante los nodos del mismo, mientras que los pasillos que las unen ser�an las aristas.\par
Por otra parte, la IA de los enemigos se obtendr� utilizando Programaci�n Gen�tica, t�cnica evolutiva que permite evolucionar programas o estrategias codificadas como expresiones.\par
Tambi�n se presenta un Framework de Programaci�n Gen�tica que permite experimentar con las t�cnicas de generaci�n de IAs, permitiendo modificar y ajustar cualquiera de los par�metros involucrados en el proceso. \par

\endinput
% Variable local para emacs, para  que encuentre el fichero maestro de
% compilaci�n y funcionen mejor algunas teclas r�pidas de AucTeX
%%%
%%% Local Variables:
%%% mode: latex
%%% TeX-master: "../Tesis.tex"
%%% End:
