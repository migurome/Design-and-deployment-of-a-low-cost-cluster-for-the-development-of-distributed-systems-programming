%---------------------------------------------------------------------
%
%                          Herramientas usadas
%
%---------------------------------------------------------------------

\chapter*{Herramientas utilizadas}

\begin{FraseCelebre}
\begin{Frase}
Si la �nica herramienta que tiene es un martillo, pensar� que cada problema que surge es un clavo.
\end{Frase}
\begin{Fuente}
Mark Twain, Escritor y humorista estadounidense. 	
\end{Fuente}
\end{FraseCelebre}

%\begin{resumen}
%...
%\end{resumen}

\section*{Visual Studio}
Desde el comienzo del proyecto, tuvimos claro que realizar�amos el proyecto en lenguaje C++. Pese a ser un lenguaje m�s complejo, tiene la ventaja de ser m�s eficiente y muy utilizado para la implementaci�n de videojuegos. Adem�s, existen gran cantidad de librer�as que pueden ser incluidas de forma f�cil en un proyecto. Por este motivo, decidimos utilizar Visual Studio 2013 como IDE con la que desarrollar todo el proyecto.
\section*{SFML}
SFML es una librer�a desarrollada para C++, especialmente utilizada en la creaci�n de videojuegos 2D. Permite el uso de elementos gr�ficos de forma sencilla y cuenta con peque�os m�dulos para el manejo de sonidos y conexiones de red.\\
Con ella hemos podido realizar tanto las interfaces gr�ficas (pese a no estar directamente dise�ada para ello), como el juego final.
\section*{Github}
Para realizar el trabajo de forma colaborativa, utilizamos un repositorio privado en la plataforma GitHub, lo que nos permite llevar un control de versiones del proyecto, dividirlo en ramas en funci�n de la fase del desarrollo y mantenernos informados de los cambios que ha introducido cada miembro del grupo.
\section*{Latex}
\LaTeX\ es un sistema de preparaci�n de documentos. Est� orientado a la presentaci�n de escritos que requieran de calidad profesional. Se compone de una serie de macros que ayudan a usar el lenguaje \TeX\ \citep*{Tex}. Permite,a su vez, separar el contenido del formato del documento. En este trabajo, hemos utilizado \LaTeX\ para la maquetaci�n de la memoria.
\section*{Mendeley}
\textit{Mendeley} es una aplicaci�n utilizada para llevar la gesti�n de las referencias bibliogr�ficas. Permite a�adir documentos gestionando la propia aplicaci�n los datos de la referencia o a�adir las referencias manualmente, teniendo que meter todos los datos de la referencia. Puede utilizarse para toda clase de art�culos, adem�s de para p�ginas web. Tambi�n permite el uso de varios formatos de bibliograf�a diferentes, dependiendo de lo que se quiera mostrar en la referencia.

\restauraCabecera

% Variable local para emacs, para  que encuentre el fichero maestro de
% compilaci�n y funcionen mejor algunas teclas r�pidas de AucTeX
%%%
%%% Local Variables:
%%% mode: latex
%%% TeX-master: "../Tesis.tex"
%%% End:
