%---------------------------------------------------------------------
%
%                          Parte 1
%
%---------------------------------------------------------------------
%
% Parte1.tex
% Copyright 2009 Marco Antonio Gomez-Martin, Pedro Pablo Gomez-Martin
%
% This file belongs to the TeXiS manual, a LaTeX template for writting
% Thesis and other documents. The complete last TeXiS package can
% be obtained from http://gaia.fdi.ucm.es/projects/texis/
%
% Although the TeXiS template itself is distributed under the 
% conditions of the LaTeX Project Public License
% (http://www.latex-project.org/lppl.txt), the manual content
% uses the CC-BY-SA license that stays that you are free:
%
%    - to share & to copy, distribute and transmit the work
%    - to remix and to adapt the work
%
% under the following conditions:
%
%    - Attribution: you must attribute the work in the manner
%      specified by the author or licensor (but not in any way that
%      suggests that they endorse you or your use of the work).
%    - Share Alike: if you alter, transform, or build upon this
%      work, you may distribute the resulting work only under the
%      same, similar or a compatible license.
%
% The complete license is available in
% http://creativecommons.org/licenses/by-sa/3.0/legalcode
%
%---------------------------------------------------------------------

% Definici�n de la primera parte del manual

\partTitle{Conceptos b�sicos}

\partDesc{Esta primera parte del manual presenta los conceptos b�sicos
  de \texis. Contiene un cap�tulo de introducci�n, seguido de una
  descripci�n de la estructura de \texis\ y c�mo se genera el
  documento final, para terminar con un cap�tulo en el que se describe
  el proceso de edici�n sugerido y los comandos que \texis\
  proporciona para facilitar dicho proceso.}

\partBackText{En realidad la divisi�n por partes del manual no aporta
  demasiado al lector; se ha dividido en varias partes debido a que,
  en la pr�ctica, el c�digo de este manual sirve como ejemplo de uso
  de \texis.

  En un contexto distinto, es posible que un manual de este tipo no
  habr�a tenido estas partes as� de diferenciadas.}

\makepart
