%---------------------------------------------------------------------
%
%                          Herramientas usadas
%
%---------------------------------------------------------------------

\chapter*{Herramientas utilizadas}


%\begin{resumen}
%...
%\end{resumen}

\section*{Github}
Para realizar el trabajo de forma colaborativa, utilizamos un repositorio privado en la plataforma GitHub, lo que nos permite llevar un control de versiones del proyecto, dividirlo en ramas en funci�n de la fase del desarrollo y mantenernos informados de los cambios que ha introducido cada miembro del grupo.
\section*{Latex}
\LaTeX\ es un sistema de preparaci�n de documentos. Est� orientado a la presentaci�n de escritos que requieran de calidad profesional. Se compone de una serie de macros que ayudan a usar el lenguaje \TeX\ \citep*{Tex}. Permite,a su vez, separar el contenido del formato del documento. En este trabajo, hemos utilizado \LaTeX\ para la maquetaci�n de la memoria.

\restauraCabecera

% Variable local para emacs, para  que encuentre el fichero maestro de
% compilaci�n y funcionen mejor algunas teclas r�pidas de AucTeX
%%%
%%% Local Variables:
%%% mode: latex
%%% TeX-master: "../Tesis.tex"
%%% End:
