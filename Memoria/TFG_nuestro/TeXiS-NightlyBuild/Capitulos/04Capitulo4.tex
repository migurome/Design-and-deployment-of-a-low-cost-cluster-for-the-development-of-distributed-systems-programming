%---------------------------------------------------------------------
%
%                          Cap�tulo 4
%
%---------------------------------------------------------------------
\chapter{Configuraci\'on del cluster}


\section{Virtualizaci\'on del Sistema}
Debido al volumen del material necesario para el desarrollo del proyecto, la necesidad de realizar el montaje y desmontaje de forma manual de �ste y la dificultad en el acceso y configuraci�n de cada uno de los nodos que lo componen decidimos invertir tiempo en realizar una virtualizaci�n del sistema para minimizar los problemas antes descritos. As�, al disponer de un entorno virtual, que replica el cluster real, se minimiza el tiempo necesario para la configuraci�n de los distintos servicios y servidores del sistema operativo y sirve como banco de pruebas para el desarrollo software.\par

Para ello hemos utilizamos VMware workstation 12 como plataforma de software de virtualizaci�n y una ISO de Raspbian basada en Debian Stretch disponible en la p�gina oficial de Raspberry Pi. Aunque el sistema del cluster parte de una versi�n diferente de debian el sistema de carpetas y sobre todo la instalaci�n de SIMCAN es similar al del entorno real.\par

En el repositorio en github existe una r�plica del sistema de carpetas tanto para el servidor como para los nodos slave. Cada una de las carpetas y ficheros modificados en el sistema durante la configuraci�n del sistema quedan reflejados en el repositorio, disponiendo as� de un listado en forma de �rbol de todas las configuraciones necesarias para el correcto funcionamiento de este.\par

\figurahere{Bitmap/2}{width=.7\textwidth}{fig:2}%
{Estructura en �rbol}
\subsection{Raspbian y sus vicisitudes}

\section{Montaje de servidores}
\subsection{DHCP}
\subsection{NFS}
\subsection{SSH}
\section{Instalaci\'on de SIMCAN}

Antes de poder instalar el software SIMCAN es necesario realizar la instalaci�n previa del simulador modular de eventos discretos de redes Omnet++ en su versi�n 4.6. Adem�s de la suite Inet, que implementa modelos de c�digo abierto OMNeT++ para redes cableadas, inal�mbricas y m�viles.\par

Debido a la baja potencia de la Raspberry esta no es capaz de lanzar la aplicaci�n de forma gr�fica, esto supone un problema a la hora de realizar la instalaci�n del software. Es por esto que todas las instalaciones han de realizarse de forma manual a trav�s del terminal.\par

Esto afecta principalmente a la instalaci�n de Inet, ya que las principales gu�as de instalaci�n disponibles en las webs oficiales parten siempre del entorno gr�fico de Omnet++.\par

Durante el desarrollo del proyecto hemos generado unas gu�as de instalaci�n y configuraci�n paso a paso que se desglosar�n el el siguiente apartado.\par

\subsection{Gu\'ia paso a paso de instalaci\'on en arquitectura ARM}

\begin{itemize}
\item Descargar los tar.gz de Omnet 4.6, Inet, simcan.tar.
\item Esta �ltima (simcan) incluye las bibliotecas que se necesitan para la compilaci�n.
\item Copia los archivos .tar de Omnet e Inet en /pi y se descomprimen
\item Desde el directorio /pi ejecuta los siguientes comandos
\end{itemize}



\section{Optimizaci\'on y rendimiento}
\subsection{Modificaci\'on del GRUB}
\subsection{Paralelizaci\'on del arranque en servidor}


\section{Seguridad}
\subsection{Eliminaci\'on de usuarios y permisos}

\section{Inicializaci\'on del sistema mediante scripts}


% Variable local para emacs, para  que encuentre el fichero maestro de
% compilaci�n y funcionen mejor algunas teclas r�pidas de AucTeX
%%%
%%% Local Variables:
%%% mode: latex
%%% TeX-master: "../Tesis.tex"
%%% End:
