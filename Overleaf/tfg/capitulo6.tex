\newpage

\thispagestyle{empty}
\mbox{}

\chapter{Cliente Servidor en JAVA}

\label{ch:capitulo6.tex}

\begin{FraseCelebre}
	\begin{Frase}
		'long long long' is too long for GCC
	\end{Frase}
	\begin{Fuente}
	Some GCC programer
	\end{Fuente}
\end{FraseCelebre}

La introducción que creamos necesaria para este capítulo

\section{Creación de un .jar}
\label{makereference6.2}
\paragraph{}

Para poder ejecutar los \textit{.jar} correctamente debemos tener nuestra versión de \textit{Java} en la \textbf{versión 1.8}, esto se configura de la siguiente manera:

\begin{lstlisting}[language=c,frame=single,numbers=none]
	1 sudo update-alternatives --config java
	2 Elegiremos la opción /usr/lib/jvm/oracle-java8-jdk-i386/jre/bin/java
	3 En caso de que no se encuentre:
		3.1	sudo apt-get upgrade
		3.2	sudo apt-get update
		3.3	sudo apt-get install default-jre
	Dentro de eclipse:
	4 Seleccionamos: Proyect - export - runnable JAR file
		4.2	launch configuration elegimos la clase que queremos hacer ejecutable (client.java o server.java)
		4.3 opiones seleccionadas: 
        	4.3.1 extract required libraries into generated jar
		4.4 finish
    
\end{lstlisting}


	
	 


