\newpage
\begin{flushleft}
{\bf \Huge Abstract}
\end{flushleft}
\vspace{1cm}
\setlength{\baselineskip}{0.8cm}

\begin{FraseCelebre}
	\begin{Frase}
		¡Tranquilos, es sólo un nombre! Como la Zona de la Muerte o la Zona sin Retorno. Esos nombres son normales en la Galaxia del Terror.
	\end{Frase}
	\begin{Fuente}
	Profesor Hubert Farnsworth
	\end{Fuente}
\end{FraseCelebre}

The main objective of this work is the development of a low budget computing cluster. For this we have made use of the so-called Raspberry Pi reduced-board computers, as processing nodes on a Debian Jessie Linux distribution.


We set out the following specific objectives: design of the container, distribution of each of the elements within it, study of the temperatures and behavior of the hardware under stress situations, deployment and development of the system software, performance study of the same, generation of installation guides and specific aspects of system configuration, development of a task planner for the distribution of work between the nodes and development of software to send and receive jobs to the cluster.

Throughout the document, each of the previous points will be developed. This work is intended to complement the laboratory practices that are carried out in the subject of Distributed Systems Programming (PSD) in the Computing Faculty of the Complutense University of Madrid. 

At the end of the project, the cluster will allow to students the possibility of performing some of the computation tasks on it, will also serve as an example of a real distributed system, with which to strengthen the knowledge of the subject.

